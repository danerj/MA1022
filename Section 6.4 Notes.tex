\documentclass[11pt]{article}

\usepackage{amssymb}
\usepackage{amsmath}
\usepackage{amsthm}
\usepackage{indentfirst}

\title{Section 6.4 Notes}

\begin{document}
\maketitle

\section*{6.4.8a}

Set up an integral for the area of the surface generated by revolving the given curve about the indicated axis.\\

$$y = \int_{1}^{x} \, \sqrt{t^2 - 1} \, dt,\quad 1\leq x \leq \sqrt{5}; \quad x - \text{axis} $$

The appropriate formula for the surface area, $S$, is:

$$S = \int_{a}^{b} 2\pi y \sqrt{1 + \left(\frac{dy}{dx}\right)^2} \, dx \,.$$

In this case, $a = 1$, $b = \sqrt{5}$, $y = \int_{1}^{x} \, \sqrt{t^2 - 1} \, dt$, and so

$$\frac{dy}{dx} = \frac{d}{dx} \int_{1}^{x} \, \sqrt{t^2 - 1} \, dt = \sqrt{x^2 - 1} \;.$$

Therefore the integral we use to calculate $S$ is:

$$S = \int_{1}^{\sqrt{5}} 2\pi  \left(\int_{1}^{x} \, \sqrt{t^2 - 1} \, dt\right) \sqrt{1 + \left(\sqrt{x^2 - 1}\right)^2} \, dx$$ $$=\int_{1}^{\sqrt{5}} 2\pi  \left(\int_{1}^{x} \, \sqrt{t^2 - 1} \, dt\right) \sqrt{1 + |x^2 - 1|} \, dx=\int_{1}^{\sqrt{5}} 2\pi  \left(\int_{1}^{x} \, \sqrt{t^2 - 1} \, dt\right) x \, dx\,.$$


Note that we used the assumption that $1\leq x \leq \sqrt{5}$ to conclude that $|x^2-1| = x^2-1$ and $\sqrt{x^2} = |x| = x$ in order to simplify in the last step. This is something you do need to check before removing absolute value bars. 

\section*{6.4.10}

Find the lateral surface area of the cone generated by revolving the line segement $y = x/2$, $0\leq x \leq 4$ abouth the $y-$axis. Check your answer with the geometry formula \\

Lateral surface area = $\frac{1}{2}  \times $ base circumference $\times$ slant height.\\

Since we are revolving our curve about the $y$-axis, the appropriate formula to calculate surface area $S$, is

$$S = \int_{c}^{d} 2\pi x \sqrt{1 + \left(\frac{dx}{dy}\right)^2} \, dy \,.$$ 

So we'll need to write $x$ as a function of $y$ and determine the bounds $c$ and $d$ for $y$ such that $c\leq y \leq d$. Here $x = 2y$ and $0 \leq y \leq 2$ (graph this curve if it's not clear why those are our bounds). So $\frac{dx}{dy} = 2$ and the lateral surface area is:

$$S = \int_{0}^{2} 2\pi (2y) \sqrt{1 + 2^2} \, dy \, = 4\sqrt{5}\pi \int_{0}^{2} y \, dy = 2\sqrt{5}\pi \left(y^2 \big\rvert_{0}^{2}\right) = 8\sqrt{5}\pi \; \text{units}^2. $$

Using the given geometry formula\\

Lateral surface area = $\frac{1}{2} \times (2\pi(4))$ $\times \sqrt{4^2 + 2^2}$ = $ 4\pi$ $\times \sqrt{20} = 8\sqrt{5}\pi \text{ units}^2.$\\

The slant height is the length of the diagonal line segment connect the points $(0,0)$ and $(4,2)$, which are the endpoints of the portion of the curve that is revolved about the $y$-axis.

\section*{6.4.16}

Find the area of the surface generated by revolving the given curve about the indicated axis. \\

$$y = \sqrt{x+1}, \quad 1\leq x \leq 5; \quad x-\text{axis}.$$

The appropriate formula for the surface area, $S$, is:

$$S = \int_{a}^{b} 2\pi y \sqrt{1 + \left(\frac{dy}{dx}\right)^2} \, dx \,.$$

Here $a= 1$, $b = 5$, and $\frac{dy}{dx} = \frac{1}{2}\frac{1}{\sqrt{x+1}}$. So the surface area, $S$, is given by:

\begin{align}
S &= \int_{1}^{5} 2\pi \sqrt{x+1} \sqrt{1 + \left(\frac{1}{2}\frac{1}{\sqrt{x+1}}\right)^2} \, dx \\
&= \int_{1}^{5} 2\pi \sqrt{x+1} \sqrt{1 + \frac{1}{4}\frac{1}{|x+1|}} \, dx\\
&= \int_{1}^{5} 2\pi \sqrt{x+1} \sqrt{1 + \frac{1}{4}\frac{1}{x+1}} \, dx\\
&= \int_{1}^{5} 2\pi \sqrt{x+1 + \frac{1}{4}\frac{x+1}{x+1}} \, dx\\
&= \int_{1}^{5} 2\pi \sqrt{x+\frac{5}{4}} \, dx\\
&=2\pi \frac{2}{3}\left((x+\frac{5}{4})^{3/2}\big\rvert_{1}^{5}\right)\\
&=\frac{4\pi}{3}\left((25/4)^{3/2} - (9/4)^{3/2}\right)\\
&=\frac{4\pi}{3}\left(\frac{125}{8} - \frac{27}{8}\right)\\
&=\frac{49\pi}{3} \;.
\end{align}

To get from line 3 to line 4 we used the fact that $\sqrt{a}\sqrt{b} = \sqrt{ab}$ for $a,b\geq 0$. 

\end{document}